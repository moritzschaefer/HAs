\documentclass[10pt,a4paper,german,landscape]{article} \usepackage[utf8]{inputenc} % damit man im Text äöü verwenden kann

\usepackage{amsmath} \usepackage{amsfonts} \usepackage{amssymb} \usepackage{graphicx} \usepackage{paralist} \usepackage{etoolbox}

\usepackage[ngerman]{babel}
\usepackage{geometry}
\geometry{verbose,a4paper,tmargin=25mm,bmargin=25mm,lmargin=15mm,rmargin=20mm}
\usepackage[arrow, matrix, curve]{xy}


\usepackage{semantic}
\usepackage{xifthen} \usepackage{ifthen} \usepackage{ifmtarg} %\usepackage{graphicx} %wenn man die diagrame lieber so einscannt

\usepackage{paralist} % mehr möglichkeiten bei Aufzählungen

\usepackage{tikz} % tiks bedeutet "tikz ist kein Zeichenprogramm" % wird für die grafiken benötigt

\usepackage{tikz-cd} %Create commutative diagrams with TikZ (http://www.ctan.org/pkg/tikz-cd) \usetikzlibrary{matrix,arrows,decorations.pathmorphing} %kp, copy-pasted

\RequirePackage{etoolbox}

\RequirePackage{ifthen}

\RequirePackage{xifthen}

\RequirePackage{ifmtarg} %TODO: Wie kann ich die TheGi Macros einbinden?

\usepackage{Macros} %TheGi Macros (http://stackoverflow.com/questions/5993965/include-sty-file-from-a-super-subdirectory-of-main-tex-file) %\usepackage{bussproofs}



\begin{document}

\title{TheGI III \\
1. Abgabe\\
Tutor:Nadine Rohde Raum: E-N 187 Uhrzeit: Mittwoch 12-14 Uhr \\ } % Declares the document's title.
\author{Moritz Schäfer (350651), \\
Marco Morik (348689), Max Gotthardt(350464)} % Declares the author's name.
\date{May 13 2014} % Deleting this command produces today's date.

\maketitle



\newpage

\section*{1. Blatt (Hausaufgabe 2)} % Produces section heading. Lower-level
\subsection*{Aufgabe 1.a)}

% aufzaehlung enumerate item
\begin{itemize}
  \item $ Studentengruppe \CCSDef \In{betreten}{}.\In{aktiv}{}.\Out{erstellen}{}.\In{verlassen}{}.Studentengruppe $
  \item
    $ HAGruppe \CCSDef \In{betreten}{}.\In{reden}{}.\Out{schlechteHas}{}.\In{verlassen}{}.HAGruppe $ \\
    $ TutGruppe \CCSDef \In{betreten}{}.\In{zuhoeren}{}.\Out{guteHAs}{}.\In{verlassen}{}.TutGruppe $

  \item
$ Tutor \CCSDef \In{betreten}{}.\In{verlassen}{}.Tutor $
\item
  $ Raum \CCSDef TutorienRaum + HARaum \\
TutorienRaum \CCSDef \Out{tutorBetreten}{}.\Out{tutGruppeBetreten}{}.\Out{zuhoeren}{}.\Out{tutGruppeVerlassen}{}.\Out{tutorVerlassen}{}.Raum \\
HARaum \CCSDef \Out{haGruppeBetreten}{}.\Out{reden}{}.\Out{haGruppeVerlassen}{}.Raum $
\item

  $ Uni \CCSDef  \Res {(((Raum | \Relabel {Tutor}{tutorBetreten/betreten,tutorVerlassen/verlassen}) | \\ \Relabel {TutGruppe}{tutGruppeBetreten/betreten,tutGruppeVerlassen/verlassen,erstellen/guteHAs} )| \\ \Relabel {HAGruppe}{haGruppeBetreten/betreten,haGruppeVerlassen/verlassen,erstellen/schlechteHAs})\\}{tutGruppeBetreten,tutGruppeVerlasse,haGruppeBetreten,haGruppeVerlassen,reden,zuhoeren,tutorBetreten,tutorVerlassen}  $
\end{itemize}

\subsection*{Aufgabe 1.b)}

	Zum Kurzfassen:\\
	$A := \{tutGruppeBetreten,tutGruppeVerlasse,haGruppeBetreten,haGruppeVerlassen,reden,zuhoeren,tutorBetreten,tutorVerlassen\}$\\ \\
	$HAGruppe \CCSDef \Relabel {\In{betreten}{}.\In{reden}{}.\Out{schlechteHas}{}.\In{verlassen}{}.HAGruppe}{haGruppeBetreten/betreten,haGruppeVerlassen/verlassen,erstellen/schlechteHAs} $ \\
	$HAGruppe' \CCSDef \Relabel {\In{reden}{}.\Out{schlechteHas}{}.\In{verlassen}{}.HAGruppe}{haGruppeBetreten/betreten,haGruppeVerlassen/verlassen,erstellen/schlechteHAs} $ \\
	$HAGruppe'' \CCSDef \Relabel {\Out{schlechteHas}{}.\In{verlassen}{}.HAGruppe}{haGruppeBetreten/betreten,haGruppeVerlassen/verlassen,erstellen/schlechteHAs} $ \\
	$HAGruppe''' \CCSDef \Relabel {\In{verlassen}{}.HAGruppe}{haGruppeBetreten/betreten,haGruppeVerlassen/verlassen,erstellen/schlechteHAs} $ \\ \\
	$TutGruppe \CCSDef \Relabel {\In{betreten}{}.\In{zuhoeren}{}.\Out{guteHAs}{}.\In{verlassen}{}.TutGruppe}{tutGruppeBetreten/betreten,tutGruppeVerlassen/verlassen,erstellen/guteHAs}$\\
	$TutGruppe' \CCSDef \Relabel {\In{zuhoeren}{}.\Out{guteHAs}{}.\In{verlassen}{}.TutGruppe}{tutGruppeBetreten/betreten,tutGruppeVerlassen/verlassen,erstellen/guteHAs}$\\
	$TutGruppe'' \CCSDef \Relabel {\Out{guteHAs}{}.\In{verlassen}{}.TutGruppe}{tutGruppeBetreten/betreten,tutGruppeVerlassen/verlassen,erstellen/guteHAs}$\\
	$TutGruppe''' \CCSDef \Relabel {\In{verlassen}{}.TutGruppe}{tutGruppeBetreten/betreten,tutGruppeVerlassen/verlassen,erstellen/guteHAs}$ \\ \\
	$Tutor \CCSDef 	 \Relabel {\In{betreten}{}.\In{verlassen}{}.Tutor }{tutorBetreten/betreten,tutorVerlassen/verlassen}$ \\
	$Tutor' \CCSDef 	 \Relabel {\In{verlassen}{}.Tutor }{tutorBetreten/betreten,tutorVerlassen/verlassen}$ \\ \\
$
TutorienRaum \CCSDef \Out{tutorBetreten}{}.\Out{tutGruppeBetreten}{}.\Out{zuhoeren}{}.\Out{tutGruppeVerlassen}{}.\Out{tutorVerlassen}{}.Raum \\	
TutorienRaum' \CCSDef \Out{tutGruppeBetreten}{}.\Out{zuhoeren}{}.\Out{tutGruppeVerlassen}{}.\Out{tutorVerlassen}{}.Raum \\	
TutorienRaum'' \CCSDef \Out{zuhoeren}{}.\Out{tutGruppeVerlassen}{}.\Out{tutorVerlassen}{}.Raum \\	
TutorienRaum''' \CCSDef \Out{tutGruppeVerlassen}{}.\Out{tutorVerlassen}{}.Raum \\	
TutorienRaum'''' \CCSDef \Out{tutorVerlassen}{}.Raum \\ \\	
$	
$
HARaum \CCSDef \Out{haGruppeBetreten}{}.\Out{reden}{}.\Out{haGruppeVerlassen}{}.Raum \\
HARaum' \CCSDef \Out{reden}{}.\Out{haGruppeVerlassen}{}.Raum \\
HARaum'' \CCSDef \Out{haGruppeVerlassen}{}.Raum \\
$

  \begin{minipage}{0.4\linewidth-7.112pt}
   \makebox[\linewidth]{
    \begin{xy}
      \xymatrix
      {
      & \Pro{Uni} \ar[dr]^{\In{\tau}{}} \ar[dl]^{\In{\tau}{}} &  \\
      \Pro{(TutorienRaum' |Tutor' |TutGruppe |HAGruppe) \backslash A  } \ar[d]^{\In{\tau}{}} & & \Pro{ (HARaum'|Tutor|TutGruppe|HAGruppe') \backslash A} \ar[d]^{\In{\tau}{}}\\
      \Pro{(TutorienRaum'' |Tutor' |TutGruppe' |HAGruppe) \backslash A  }\ar[d]^{\In{\tau}{}}& & \Pro{ (HARaum''|Tutor|TutGruppe|HAGruppe'') \backslash A} \ar[d]^{\Out{erstellen}{}} \\
      \Pro{(TutorienRaum''' |Tutor' |TutGruppe'' |HAGruppe) \backslash A  }\ar[d]^{\Out{erstellen}{}}& & \Pro{ (HARaum''|Tutor|TutGruppe|HAGruppe''') \backslash A} \ar @/^ 40px/ [luuu]^{\In{\tau}{}}  \\
      \Pro{(TutorienRaum''' |Tutor' |TutGruppe''' |HAGruppe) \backslash A  }\ar[d]^{\In{\tau}{}}& & \\
      \Pro{(TutorienRaum'''' |Tutor' |TutGruppe |HAGruppe) \backslash A  } \ar `/20pt[ru] [ruuuuu]_{\In{\tau}{}}& & \\
      }
    \end{xy}
   }
  \end{minipage}
  %

\subsection*{Aufgabe 2.a)}
$(\Proc, \Act, \{\CCSTrans{a} | a \in \Act\})$
mit: \\
$\Proc = \{T, T_i | i \in [1,8] \}$ \\
$\Act = \{\In{a}{}, \Out{a}{}, \In{b}{}, \Out{b}{}, \In{\tau}{} \}$ \\
$
\CCSTrans{\In{a}{}} = \{(T,T_1), (T,T_5), (T_4,T_6) \} \\
\CCSTrans{\Out{a}{}} = \{(T,T_4), (T_5,T_6), (T_7,T_8) \} \\
\CCSTrans{\In{b}{}} = \{(T_1,T_2), (T_3,T_8), (T,T_4), (T_5,T_6),(T_7,T_8) \} \\
\CCSTrans{\Out{b}{}} = \{(T_1,T_3), (T_2,T_8), (T_5,T_7), (T_6,T_8) \} \\
\CCSTrans{\In{\tau}{}} = \{(T,T_6), (T_1,T_8), (T_5,T_8) \} \\
$

\subsection*{Aufgabe 2.b)}
$
T \CCSDef (((\In{a}{}.T_1 + \Out{a}{}.T_4) + \In{b}{}.T_4) + \In{a}{}.T_5) + \In{\tau}{}.T_6 \\
T_1 \CCSDef \In{b}{}.\nil | \Out{b}{}.\nil \\
T_4 \CCSDef \In{a}{}.T_6 \\
T_5 \CCSDef (\Out{a}{}.\nil + \In{b}{}.\nil) | \Out{b}{}.\nil \\
T_6 \CCSDef \Out{b}{}.\nil
$
\subsection*{Aufgabe 2.c)}
$T \CCSDef (\In{a}{}.\Out{b}{}.\nil | (\In{b}{}.\nil + \Out{a}{}.\nil )) $

\subsection*{Aufgabe 3.a)}
\begin{displaymath}
  \inference[REL]
  {
    \inference[RES]
    {
      \inference[COM2]
      {
        \inference[ACT]
        {
        }
        {
          \Out{b}{}.A \CCSTrans{\Out{b}{}} A
        }
      }
      {
        B | \Out{b}{}.A \CCSTrans{\Out{b}{}} B | A
      }
    }
    {
      (B | \Out{b}{}.A) \setminus \{a\} \CCSTrans{\Out{b}{}} ( B | A ) \setminus \{a\}
    }
  }
  {
    (\Relabel{(B | \Out{b}{}.A) \setminus \{a\})}{a/b} \CCSTrans{\Out{a}{}} \Relabel{ ( B | A ) \setminus \{a\} }{a/b}
  }
\end{displaymath}
\subsection*{Aufgabe 3.b)}
Es gibt keine möglichen Aktionen da nach Anwendung der Relabeling COM3 nicht mehr funktioniert da die eine Aktion $a$ nicht mit $\Out{b}{}$ kombiniert werden kann.
\subsection*{Aufgabe 3.c)}
\begin{displaymath}
  \inference[RES]
  {
    \inference[COM3]
    {
      \inference[CON]
      {
        \inference[ACT]
        {
        }
        {
          \In{a}{}.A \CCSTrans{\In{a}{}} A
        }
      }
      {
        B \CCSTrans{\In{a}{}} A
      }
      &
      \inference[REL]
      {
        \inference[ACT]
        {}
        {
          \Out{b}{}.A \CCSTrans{\Out{b}{}} A
        }
      }
      {
        \Relabel{\Out{b}{}.A}{a/b} \CCSTrans{\Out{a}{}} \Relabel{ A }{a/b}
      }
    }
    {
      B | \Relabel{(\Out{b}{}.A)}{a/b} \CCSTrans{\tau} A | \Relabel{ A }{a/b}
    }
  }
  {
    (B | \Relabel{(\Out{b}{}.A)}{a/b}) \setminus \{a\} \CCSTrans{\tau} ( A | \Relabel{ A }{a/b})\setminus \{a\}
  }
\end{displaymath}


\newpage
\section*{2. Blatt (Hausaufgabe 3)}
\subsection*{Aufgabe 4.a)}
Gilt.\\
Sei $ \MC {B}=\{ (P|Q,Q|P)\} \cup Id_{Proc} $\\
Wir zeigen, dass \MC {B} eine starke Bisimulation ist.
\begin{itemize}
\item Betrachte $Id_{Proc} \subseteq \MC {B}$: 
\subitem - $Id_{Proc}$ ist per Definition eine Bisimulation 
\item Betrachte $ (P|Q ,Q|P) \in \MC{B}$
\subitem - Transitionen von P$|$Q: \\
1. Fall COM1
\begin{displaymath}
  \inference[COM1]
  {
  \inference[ACT]
  {}{P \CCSTrans {\alpha} P'}
  }{P | Q  \CCSTrans {\alpha} P' | Q }
\end{displaymath}
\\
Wähle 
\begin{displaymath}
  \inference[COM2]
  {
  \inference[ACT]
  {}{P \CCSTrans {\alpha} P'}
  }{Q | P  \CCSTrans {\alpha} Q | P' }
\end{displaymath}
  und  $ (P'|Q ,Q|P')  \in \MC{B}$ \\ \\
  2. Fall COM2
  \begin{displaymath}
  \inference[COM2]
  {
  \inference[ACT]
  {}{Q \CCSTrans {\alpha} Q'}
  }{P | Q  \CCSTrans {\alpha} P | Q' }
\end{displaymath} \\
Wähle \begin{displaymath}
  \inference[COM1]
  {
  \inference[ACT]
  {}{Q \CCSTrans {\alpha} Q'}
  }{Q | P  \CCSTrans {\alpha} Q' | P }
\end{displaymath}
	und  $ (P|Q' ,Q'|P)  \in \MC{B}$ \\ \\
	\item Da \MC {B} eine starke Bisimulation ist und $(P|Q,Q|P) \in $ \MC {B}, können wir folgern, dass $P|Q \bisim Q|P$.
\end{itemize}
\subsection*{Aufgabe 4.b)}
Gilt nicht; Gegenbeispiel: $P \CCSDef a.\nil$ \\
In diesem Fall kann $(\Out{a}{}.\nil | a.P) \backslash \{a\}$ nach einem $\tau$-Schritt nichts weiteres machen,
$\tau.P$ jedoch kann einen $\tau$-Schritt und einen a-Schritt machen.\\
Für den Fall, dass P kein a bzw. $\Out {a}{}$ enthält, gilt die Bisimulation.
\subsection*{Aufgabe 5:}
\subsubsection*{a)}
q1 wird von p3 stark simuliert:\\
Relation : \MC {P}=${(p3,q1),(p4,q2),(p5,q3),(p1,q4),(p2,q7),(p11,q6),(p10,q5),(p6,q12),(p7,q13),(p9,q11),(p8,q10),(p9,q9),(p8,q8)}$
\subsubsection*{b)}
p3 wird von q1 stark simuliert:\\
Relation : \MC {Q}=${(q1,p3),(q2,p4),(q3,p5),(q4,p1),(q7,p2),(q6,p11),(q5,p10),(q12,p6),(q13,p7),(q11,p9),(q10,p8),(q9,p9),(q8,p8)}$
\subsubsection*{c)}
p3 und q1 simulieren sich gegenseitig stark. (Siehe Teil a) und Teil b)) \\
Die Relation ist schon gegeben mit \MC {P} und \MC {Q}
\subsubsection*{d)}
p3 und q1 sind stark bisimilar, mit \MC {B}=\MC {P}$\cup$\MC {Q}. 
\subsubsection*{e)}
q12 und p6 sind ebenfalls stark bisimilar.\\
Gegeben sei \MC {B}=$\{(p6,q12),(p7,q13),(p9,q11),(p8,q10),(p9,q9),(p8,q8)\}$\\
zZ: B ist eine starke Bisimulation:\\
\begin{itemize}
\item Betrachte $(p6,q12) \in $\MC {B}:
\subitem Transition von p6:\\
	Für $p6 \CCSTrans {a} p7$ wähle $q12 \CCSTrans {a} q13$ und $(p7,q13)\in$\MC {B}\\
	Für $p6 \CCSTrans {c} p9$ wähle $q12 \CCSTrans {c} q11$ und $(p9,q11)\in$\MC {B}
\subitem Transition von q12:\\
	Für $q12 \CCSTrans {a} q13$ wähle $p6 \CCSTrans {a} p7$ und $(p7,q13)\in$\MC {B}\\
	Für $q12 \CCSTrans {c} q11$ wähle $p6 \CCSTrans {c} p9$ und $(p9,q11)\in$\MC {B}
\item Betrachte $(p7,q13) \in $\MC {B}:
\subitem Transition von p7:\\
	Für $p7 \CCSTrans {c} p6$ wähle $q13 \CCSTrans {c} q12$ und $(p6,q12)\in$\MC {B}
\subitem Transition von q13:\\
	Für $q13 \CCSTrans {c} q12$ wähle $p7 \CCSTrans {c} p6$ und $(p6,q12)\in$\MC {B}
\item Betrachte $(p9,q11) \in $\MC {B}:
\subitem Transition von p7: \\
	Für $p9 \CCSTrans {b} p8$ wähle $q11 \CCSTrans {b} q10$ und $(p8,q10)\in$\MC {B}
\subitem Transition von q13:\\
	Für $q11 \CCSTrans {b} q10$ wähle $p9 \CCSTrans {b} p8$ und $(p8,q10)\in$\MC {B}
\item Betrachte $(p8,q10) \in $\MC {B}:
\subitem Transition von p7:\\
	Für $p8 \CCSTrans {b} p9$ wähle $q10 \CCSTrans {b} q9$ und $(p9,q9)\in$\MC {B}
\subitem Transition von q13:\\
	Für $q10 \CCSTrans {b} q9$ wähle $p8 \CCSTrans {b} p9$ und $(p9,q9)\in$\MC {B}
\item Betrachte $(p9,q9) \in $\MC {B}:
\subitem Transition von p7:\\
	Für $p9 \CCSTrans {b} p8$ wähle $q9 \CCSTrans {b} q8$ und $(p8,q8)\in$\MC {B}
\subitem Transition von q13:\\
	Für $q9 \CCSTrans {b} q8$ wähle $p9 \CCSTrans {b} p8$ und $(p8,q8)\in$\MC {B}
\item Betrachte $(p8,q8) \in $\MC {B}:
\subitem Transition von p7:\\
	Für $p8 \CCSTrans {b} p9$ wähle $q8 \CCSTrans {b} q11$ und $(p9,q11)\in$\MC {B}
\subitem Transition von q13:\\
	Für $q8 \CCSTrans {b} q11$ wähle $p8 \CCSTrans {b} p9$ und $(p9,q11)\in$\MC {B}\\ \\
	Da \MC {B} eine starke Bisimulation ist und (q12,p6) $\in$ \MC {B}, können wir folgern, dass $q12 \bisim p6$.
\end{itemize}
\subsection*{Aufgabe 6:}
\subsubsection*{a)}
q1 wird von p1 stark simuliert.
Sei die Relation \MC {Q}=$\{(q1,p1),(q2,p3),(q3,p4),(q4,p5)\}$
\begin{itemize}
\item Betrachte (q1,p1)$\in$ \MC {Q}
\subitem Transition von q1:\\
Für $q1 \CCSTrans {a} q2$ wähle $p1 \CCSTrans {a} p3$ und (q2,p3) $\in$ \MC {Q}
\item Betrachte (q2,p3)$\in$ \MC {Q}
\subitem Transition von q2:\\
Für $q2 \CCSTrans {c} q3$ wähle $p3 \CCSTrans {c} p4$ und (q3,p4) $\in$ \MC {Q}\\
Für $q2 \CCSTrans {b} q4$ wähle $p3 \CCSTrans {b} p5$ und (q4,p5) $\in$ \MC {Q}
\item Betrachte (q3,p4)$\in$ \MC {Q}
\subitem Transition von q3:\\
Für $q3 \CCSTrans {a} q1$ wähle $p4 \CCSTrans {a} p1$ und (q1,p1) $\in$ \MC {Q}
\item Betrachte (q4,p5)$\in$ \MC {Q}
\subitem Beides Blätter, habe keine Transitionen. 
\end{itemize}
\MC {Q} ist die Relation die q1 mit p1 stark simuliert.
\subsubsection*{b)}
p1 wird von q1 stark simuliert.
Sei die Relation \MC {P}=$\{(p1,q1),(p2,q2),(p3,q2),(p4,q3),(p5,q4),(p5,q3)\}$
\begin{itemize}
\item Betrachte (p1,q1)$\in$ \MC {P}
\subitem Transition von p1:\\
Für $p1 \CCSTrans {a} p2$ wähle $q1 \CCSTrans {a} q2$ und (p2,q2) $\in$ \MC {P}
Für $p1 \CCSTrans {a} p3$ wähle $q1 \CCSTrans {a} q2$ und (p3,q2) $\in$ \MC {P}

\item Betrachte (p2,q2)$\in$ \MC {P}
\subitem Transition von p2:\\
Für $p2 \CCSTrans {c} p5$ wähle $q2 \CCSTrans {c} q3$ und (p5,q3) $\in$ \MC {P}

\item Betrachte (p3,q2)$\in$ \MC {P}
\subitem Transition von p3:\\
Für $p3 \CCSTrans {c} p4$ wähle $q2 \CCSTrans {c} q3$ und (p4,q3) $\in$ \MC {P}\\
Für $p3 \CCSTrans {b} p5$ wähle $q2 \CCSTrans {c} q4$ und (p5,q4) $\in$ \MC {P}

\item Betrachte (p5,q3)$\in$ \MC {P}
\subitem p5 ist ein Blatt und hat keine weiteren Relationen.

\item Betrachte (p4,q3)$\in$ \MC {P} 
\subitem Transition von p4:\\
Für $p4 \CCSTrans {a} p1$ wähle $q3 \CCSTrans {a} q1$ und (p1,q1) $\in$ \MC {P}


\item Betrachte (p5,q4)$\in$ \MC {P}
\subitem p5 ist ein Blatt und hat keine weiteren Relationen.


\end{itemize}
\MC {P} ist die Relation die q1 mit p1 stark simuliert.

\subsubsection*{c)}
Da p1 von q1 stark simuliert wird (siehe Teil b) und q1 von p1 stark simuliert wird(siehe Teil a), simulieren sie sich gegenseitig stark.
\subsubsection*{d)}
p1 und q1 sind nicht stark bisimilar:
 Angenommen p1 und q1 sind stark bisimilar, dann existiert eine Simulation \MC {B}\\
 Mit dem Übergang $\CCSTrans {a}$ gelangt man in (p2,q2) $\in$ \MC {B}\\
 Mit dem Übergang $\CCSTrans {c}$ gelangt man von dort aus nach (p5,q3) $\in$ \MC {B}\\
 Von q3 gibt es den Übergang $q3 \CCSTrans {a} q1$, jedoch existiert kein Übergang $\CCSTrans {a}$ von p5, daher kann es die Simulation \MC {B} nicht geben.
 
\end{document}
