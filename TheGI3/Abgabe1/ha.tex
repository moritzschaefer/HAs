\documentclass[10pt,a4paper,german,landscape]{article} \usepackage[utf8]{inputenc} % damit man im Text äöü verwenden kann

\usepackage{amsmath} \usepackage{amsfonts} \usepackage{amssymb} \usepackage{graphicx} \usepackage{paralist} \usepackage{etoolbox}

\usepackage[ngerman]{babel}


\usepackage{semantic}
\usepackage{xifthen} \usepackage{ifthen} \usepackage{ifmtarg} %\usepackage{graphicx} %wenn man die diagrame lieber so einscannt

\usepackage{paralist} % mehr möglichkeiten bei Aufzählungen

\usepackage{tikz} % tiks bedeutet "tikz ist kein Zeichenprogramm" % wird für die grafiken benötigt

\usepackage{tikz-cd} %Create commutative diagrams with TikZ (http://www.ctan.org/pkg/tikz-cd) \usetikzlibrary{matrix,arrows,decorations.pathmorphing} %kp, copy-pasted

\RequirePackage{etoolbox}

\RequirePackage{ifthen}

\RequirePackage{xifthen}

\RequirePackage{ifmtarg} %TODO: Wie kann ich die TheGi Macros einbinden?

\usepackage{Macros} %TheGi Macros (http://stackoverflow.com/questions/5993965/include-sty-file-from-a-super-subdirectory-of-main-tex-file) %\usepackage{bussproofs}


\begin{document}

\title{TheGI III \\
1. Abgabe }  % Declares the document's title.
\author{Moritz Schäfer (350651), \\
Marco Morik (348689), Max Gotthardt(350464)}    % Declares the author's name.
\date{May 13 2014}   % Deleting this command produces today's date.

\maketitle

Tutor: Die Blonde mit dem Arsch Raum: E-N 187 Uhrzeit: Mittwoch 12-14 Uhr \\


\section*{1. Blatt (Hausaufgabe 2)}  % Produces section heading.  Lower-level
\subsection*{Aufgabe 1.a)}

% aufzaehlung enumerate item
\begin{itemize}
  \item $ Studentengruppe \CCSDef \In{betreten}{}.\In{aktiv}{}.\Out{erstellen}{}.\In{verlassen}{}.Studentengruppe $
  \item
    $ HAGruppe \CCSDef \In{betreten}{}.\In{reden}{}.\Out{schlechteHas}{}.\In{verlassen}{}.HAGruppe $ \\
    $ TutGruppe \CCSDef \In{betreten}{}.\In{zuhoeren}{}.\Out{guteHAs}{}.\In{verlassen}{}.TutGruppe $

  \item
$ Tutor  \CCSDef \In{betreten}{}.\In{verlassen}{}.Tutor $
\item
  $ Raum \CCSDef TutorienRaum + HARaum  \\
TutorienRaum \CCSDef \Out{tutorBetreten}{}.\Out{tutGruppeBetreten}{}.\Out{zuhoeren}{}.\Out{tutGruppeVerlassen}{}.\Out{tutorVerlassen}{}.Raum \\
HARaum \CCSDef \Out{haGruppeBetreten}{}.\Out{reden}{}.\Out{haGruppeVerlassen}{}.Raum $
\item

  $ Uni \CCSDef (Raum k \Relabel {\Relabel {Tutor}{tutorBetreten/betreten}}{tutorVerlassen/verlassen} | \\ \Relabel {\Relabel {TutGruppe}{tutGruppeBetreten/betreten}}{tutGruppeVerlassen/verlassen} | \\  \Relabel {\Relabel {HAGruppe}{haGruppeBetreten/betreten}}{haGruppeVerlassen/verlassen}) $
\end{itemize}

\subsection*{Aufgabe 1.b)}
\begin{itemize}
  \item Ask tutor
\end{itemize}

\subsection*{Aufgabe 2.a)}
$(\Proc, \Act, \{\CCSTrans{a} | a \in \Act\})$
mit: \\
$\Proc = \{T, T_i | i \in [1,8] \}$ \\
$\Act = \{\In{a}{}, \Out{a}{}, \In{b}{}, \Out{b}{}, \In{\tau}{} \}$ \\
$
\CCSTrans{\In{a}{}} = \{(T,T_1), (T,T_5), (T_4,T_6) \} \\
\CCSTrans{\Out{a}{}} = \{(T,T_4), (T_5,T_6), (T_7,T_8) \} \\
\CCSTrans{\In{b}{}} = \{(T_1,T_2), (T_3,T_8), (T,T_4), (T_5,T_6),(T_7,T_8) \} \\
\CCSTrans{\Out{b}{}} = \{(T_1,T_3), (T_2,T_8), (T_5,T_7), (T_6,T_8) \} \\
\CCSTrans{\In{\tau}{}} = \{(T,T_6), (T_1,T_8), (T_5,T_8) \} \\
$

\subsection*{Aufgabe 2.b)}
$
T \CCSDef (((\In{a}{}.T_1 + \Out{a}{}.T_4) + \In{b}{}.T_4) + \In{a}{}.T_5) + \In{\tau}{}.T_6 \\
T_1 \CCSDef \In{b}{}.\nil | \Out{b}{}.\nil  \\
T_4 \CCSDef \In{a}{}.T_6 \\
T_5 \CCSDef (\Out{a}{}.\nil + \In{b}{}.\nil) | \Out{b}{}.\nil \\
T_6 \CCSDef \Out{b}{}.\nil
$
\subsection*{Aufgabe 2.c)}
$T \CCSDef (\In{a}{}.\Out{b}{}.\nil | (\In{b}{}.\nil + \Out{a}{}.\nil )) $

\subsection*{Aufgabe 3.a)}
\begin{displaymath}
  \inference[REL]
  {
    \inference[RES]
    {
      \inference[COM2]
      {
        \inference[ACT]
        {
        }
        {
          \Out{b}{}.A \CCSTrans{\Out{b}{}} A
        }
      }
      {
        B | \Out{b}{}.A \CCSTrans{\Out{b}{}} B | A
      }
    }
    {
      (B | \Out{b}{}.A) \setminus \{a\} \CCSTrans{\Out{b}{}} ( B | A ) \setminus \{a\}
    }
  }
  {
    (\Relabel{(B | \Out{b}{}.A) \setminus \{a\})}{a/b}  \CCSTrans{\Out{a}{}} \Relabel{ ( B | A ) \setminus \{a\} }{a/b}
  }
\end{displaymath}
\subsection*{Aufgabe 3.b)}
Es gibt keine möglichen Aktionen da nach Anwendung der Relabeling COM3 nicht mehr funktioniert da die eine Aktion $a$ nicht mit $\Out{b}{}$ kombiniert werden kann.
\subsection*{Aufgabe 3.c)}
\begin{displaymath}
  \inference[RES]
  {
    \inference[COM3]
    {
      \inference[CON]
      {
        \inference[ACT]
        {
        }
        {
          \In{a}{}.A \CCSTrans{\In{a}{}} A
        }
      }
      {
        B \CCSTrans{\In{a}{}} A
      }
      &
      \inference[REL]
      {
        \inference[ACT]
        {}
        {
          \Out{b}{}.A \CCSTrans{\Out{b}{}} A
        }
      }
      {
        \Relabel{\Out{b}{}.A}{a/b} \CCSTrans{\Out{a}{}} \Relabel{ A }{a/b}
      }
    }
    {
      B | \Relabel{(\Out{b}{}.A)}{a/b} \CCSTrans{\tau}  A | \Relabel{ A }{a/b}
    }
  }
  {
    (B | \Relabel{(\Out{b}{}.A)}{a/b}) \setminus \{a\}  \CCSTrans{\tau} ( A | \Relabel{ A }{a/b})\setminus \{a\}
  }
\end{displaymath}
\section*{2. Blatt (Hausaufgabe 3)}
\subsection*{Aufgabe 1.a)}
\end{document}
