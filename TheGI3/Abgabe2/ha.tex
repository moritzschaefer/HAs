\documentclass[10pt,a4paper,german,landscape,fleqn]{article} \usepackage[utf8]{inputenc} % damit man im Text äöü verwenden kann

\usepackage{amsmath} \usepackage{amsfonts} \usepackage{amssymb} \usepackage{graphicx} \usepackage{paralist} \usepackage{etoolbox}
\setlength{\mathindent}{0pt}

\usepackage[ngerman]{babel}
\usepackage{geometry}
\geometry{verbose,a4paper,tmargin=25mm,bmargin=25mm,lmargin=15mm,rmargin=20mm}
\usepackage[arrow, matrix, curve]{xy}

\usepackage{mathtools}

\usepackage{semantic}
\usepackage{xifthen} \usepackage{ifthen} \usepackage{ifmtarg} %\usepackage{graphicx} %wenn man die diagrame lieber so einscannt

\usepackage{paralist} % mehr möglichkeiten bei Aufzählungen

\usepackage{tikz} % tiks bedeutet "tikz ist kein Zeichenprogramm" % wird für die grafiken benötigt

\usepackage{tikz-cd} %Create commutative diagrams with TikZ (http://www.ctan.org/pkg/tikz-cd) \usetikzlibrary{matrix,arrows,decorations.pathmorphing} %kp, copy-pasted

\RequirePackage{etoolbox}

\RequirePackage{ifthen}

\RequirePackage{xifthen}

\RequirePackage{ifmtarg} %TODO: Wie kann ich die TheGi Macros einbinden?

\usepackage{Macros} %TheGi Macros (http://stackoverflow.com/questions/5993965/include-sty-file-from-a-super-subdirectory-of-main-tex-file) %\usepackage{bussproofs}



\begin{document}

\title{TheGI III \\
2. Abgabe\\
Tutor:Nadine Rohde Raum: E-N 187 Uhrzeit: Mittwoch 12-14 Uhr \\ } % Declares the document's title.
\author{Moritz Schäfer (350651), \\
Marco Morik (348689), Max Gotthardt(350464)} % Declares the author's name.
\date{June 9 2014} % Deleting this command produces today's date.

\maketitle



\newpage

\section*{6. Blatt } % Produces section heading. Lower-level
\subsection*{Aufgabe 1.a)}

\begin{tikzpicture}
  \node (max) at (0,2) {$(3,3)$};

  \node (a) at (-1,1) {$(2,3)$};
  \node (b) at (1,1) {$(3,2)$};

  \node (c) at (-2,0) {$(1,3)$};
  \node (d) at (0,0) {$(2,2)$};
  \node (e) at (2,0) {$(3,1)$};

  \node (f) at (-1,-1) {$(1,2)$};
  \node (g) at (1,-1) {$(2,1)$};

  \node (min) at (0,-2) {$(1,1)$};

  \draw (min)  -- (g) -- (e) -- (b) -- (max)
    (min)  -- (f) -- (c) -- (a) -- (max)
    (g)  -- (d) -- (b)
    (f)  -- (d) -- (a);
  %\draw[preaction={draw=white, -,line width=6pt}] (a) -- (e) -- (c);
\end{tikzpicture}


\subsection*{Aufgabe 1.b)}

$ supr2 : (\mathbb N^{+} \times \mathbb N^{+} ) \times (\mathbb N^{+} \times \mathbb N^{+} ) \rightarrow (\mathbb N^{+} \times \mathbb N^{+} )$ mit \\
$ supr2((x_1,y_1),(x_2,y_2)) = (max(x_1, x_2), max(y_1, y_2)) $
$ inf2 : (\mathbb N^{+} \times \mathbb N^{+} ) \times (\mathbb N^{+} \times \mathbb N^{+} ) \rightarrow (\mathbb N^{+} \times \mathbb N^{+} )$ mit \\
$ inf2((x_1,y_1),(x_2,y_2)) = (min(x_1, x_2), min(y_1, y_2)) $

\subsection*{Aufgabe 1.c)}

$\sqcup X = (x_0, y_0).\forall x,y \in X. x_0 \ge x \wedge y_0 \ge y$ \\
$\sqcap X = (x_0, y_0).\forall x,y \in X. x_0 \le x \wedge y_0 \le y$

\subsection*{Aufgabe 1.d)}

$\top$ ist nicht definiert für die Trägermenge da es kein Element in $(\mathbb N^{+} \times \mathbb N^{+} )$ gibt so dass unter der Relation $\le_2$ alle Elemente kleiner oder gleich sind (wegen der Unendlichkeit).\\
$\bot$ ist definiert mit $\bot = (1,1)$
\subsection*{Aufgabe 1.e)}
Für einen vollständigen Verband müssen sowohl $\top$ als auch $\bot$ definiert sein. Das ist nicht der Fall, also ist es kein vollständiger Verband.
\subsection*{Aufgabe 1.f)}
Damit $f$ monoton ist muss (neben der Bedingung, dass $(D, \le_2)$ eine partielle Ordnung ist, was gilt) gelten: \\
Für alle $X, Y \in (\mathbb N^{+} \times \mathbb N^{+} )$ wenn $X \le_2 Y$ dann gilt auch $f(X) \le_2 f(Y)$. \\
Seien $X,Y$ mit $X \coloneqq (x_0, x_1)$ und $Y \coloneqq (y_0, y_1)$ beliebig aber fest.
Dies ist eine Implikation also können wir $X \le_2 Y$ als Annahme nehmen und erhalten daraus: \\
$ x0! \le y0! $(1) und $x_1^2+2 x_1 - 1 \le 2 y_1^2 + 2 y_1 -1$ \\
Zu zeigen ist damit $f(X) \le_2 f(Y)$. Mit der Definition von $\le_2$ folgt daraus:
$x_0! \le y_0 \wedge x_1^2 + 2x_1 -1 \le 2 y_1^2 + 2 y_1 -1$. Das gilt wenn beide Teile der Verundung wahr sind: \\
$x_0! \le y_0$ gilt mit unserer ersten Annahme und unserer Definitionsmenge. \\
$x_1^2 + 2x_1 -1 \le 2 y_1^2 + 2 y_1 -1$ gilt ebenfalls mit unserer zweiten Annahme und unserer Definitionsmenge. \\
Damit ist die Implikation wahr und der Beweis vollständig und $f$ damit monoton.

\subsection*{Aufgabe 2.)}

\section*{7. Blatt } % Produces section heading. Lower-level

\subsection*{Aufgabe 3.a)}
Damit $(B,\le)$ eine partielle Ordnung ist muss für die Relation $\le$ gelten:\\
Seien $f,g,h \in B$ \\
\begin{enumerate}
  \item $\le$ ist reflexiv: \\
    \begin{align}
      &\forall f \in B. f \le f\\
      &{\stackrel{\text{def $\le$}}{\Rightarrow}} f^{-1}(\{1\}) \subseteq f^{-1}(\{1\})
    \end{align}
    Gilt, da eine Menge auch Teilmenge von sich selbst ist.
  \item $\le$ ist antisymmetrisch: \\
    Sei $A=\{0,1\}^n$.
    \begin{align}
      &f \le g \wedge g \le f \rightarrow f = g \\
      &{\stackrel{\text{def $\le$}}{\Rightarrow}} f^{-1}(\{1\}) \subseteq g^{-1}(\{1\}) \wedge g^{-1}(\{1\}) \subseteq f^{-1}(\{1\}) \rightarrow f = g \\
    \end{align}
      Der vordere Teil der Implikation ist die Definition von Mengengleichheit. Wir haben also also als Annahme: $f^{-1}(\{1\}) = g^{-1}(\{1\})$ und müssen zeigen: $f=g$
    \begin{align}
      &f^{-1}(\{1\}) = g^{-1}(\{1\}) \\
      &{\stackrel{\text{def $^{-1}$}}{\Rightarrow}} \{a \in A | f(a) \in \{1\}\} = \{a \in A | g(a) \in \{1\}\} \\
      &\Rightarrow \{a \in A | f(a) = 1\} = \{a \in A | g(a) = 1\} \\
    \end{align}
    Da der Funktionswert 1 die gleiche Urbildmenge für beide Funktionen hat und die Funktion auf A vollständig definiert ist, sind die Funktionen gleich, was zu zeigen war. Also ist die Relation $\le$ antisymmetrisch.
  \item $\le$ ist transitiv: \\
    \begin{align}
      &f \le g \wedge g \le h \rightarrow f \le h \\
      &{\stackrel{\text{def $\le$}}{\Rightarrow}} f^{-1}(\{1\}) \subseteq g^{-1}(\{1\}) \wedge g^{-1}(\{1\}) \subseteq h^{-1}(\{1\}) \rightarrow f^{-1}(\{1\}) \subseteq h^{-1}(\{1\})\\
      &{\stackrel{\text{def $^{-1}$}}{\Rightarrow}} \{a \in A | f(a) \in \{1\}\} \subseteq \{a \in A | g(a) \in \{1\}\} \wedge \{a \in A | g(a) \in \{1\}\} \subseteq \{a \in A | h(a) \in \{1\}\} \rightarrow \{a \in A | f(a) \in \{1\}\} \subseteq \{a \in A | h(a) \in \{1\}\}
    \end{align}
    Das ist gegeben durch die Transitivität von Teilmengen.
\end{enumerate}
Da die Relation $\le$ auf $B$ die drei Eigenschaften hat, ist $(B,\le)$ eine partielle Ordnung.
\subsection*{Aufgabe 3.b)}
Da gegeben ist, dass $(B,\le)$ ein Verband ist muss nur noch gezeigt werden, dass
$B$ endlich ist, damit Lemma 4.4 gilt und wir einen vollständigen Verband haben. \\
Für ein gegebenes n können wir nur eine endliche Anzahl an verschiedenen Funktionen definieren. Damit ist $B$ endlich. \\
Wir können nur eine endliche Anzahl verschiedener Funktionen definieren, da die $n$ verschiedenen Variablen nur $2^n$ verschiedene Werte annehmen können. Damit existieren $2^{2^n}$ verschiede Funktionen(Da die Wertemenge nur 2 Werte enthält). Das ist eine begrenzte Anzahl an Funktionen.
\newpage

\subsection*{Aufgabe 5.)}
\begin{align}
&\MC{F}^1(\Proc \times \Proc) = s(r(\{(P_1,P_5),(P_1,_P4),(P_1,P_4),(P_2,P_6\})) \\
&\MC{F}^2(\Proc \times \Proc) = s(r(\{(P_1,P_5)\})) \\
&\MC{F}^3(\Proc \times \Proc) = \MC{F}^2 \\
\end{align}
Daher erhalten wir $P_1$ und $P_2$ als einzige nicht triviale größte Bisimulation.

Es gilt daher $P_1 \bisim P_5$
\end{document}
