\documentclass[10pt,a4paper,german,landscape,fleqn]{article} \usepackage[utf8]{inputenc} % damit man im Text äöü verwenden kann

\usepackage{amsmath} \usepackage{amsfonts} \usepackage{amssymb} \usepackage{graphicx} \usepackage{paralist} \usepackage{etoolbox}
\setlength{\mathindent}{0pt}

\usepackage[ngerman]{babel}
\usepackage{geometry}
\geometry{verbose,a4paper,tmargin=25mm,bmargin=25mm,lmargin=15mm,rmargin=20mm}
\usepackage[arrow, matrix, curve]{xy}

\usepackage{mathtools}

\usepackage{semantic}
\usepackage{xifthen} \usepackage{ifthen} \usepackage{ifmtarg} %\usepackage{graphicx} %wenn man die diagrame lieber so einscannt

\usepackage{paralist} % mehr möglichkeiten bei Aufzählungen

\usepackage{tikz} % tiks bedeutet "tikz ist kein Zeichenprogramm" % wird für die grafiken benötigt

\usepackage{tikz-cd} %Create commutative diagrams with TikZ (http://www.ctan.org/pkg/tikz-cd) \usetikzlibrary{matrix,arrows,decorations.pathmorphing} %kp, copy-pasted

\RequirePackage{etoolbox}

\RequirePackage{ifthen}

\RequirePackage{xifthen}

\RequirePackage{ifmtarg} %TODO: Wie kann ich die TheGi Macros einbinden?

\usepackage{Macros} %TheGi Macros (http://stackoverflow.com/questions/5993965/include-sty-file-from-a-super-subdirectory-of-main-tex-file) %\usepackage{bussproofs}
\newcommand{\ausws}[1]{ \langle \cdot #1 \cdot \rangle }
\newcommand{\auswl}[1]{ [ \cdot #1 \cdot ] }
\newcommand{\auswf}[1]{[\![ #1 ]\!]}


\begin{document}

\title{TheGI III \\
3. Abgabe\\
Tutor:Nadine Rohde Raum: E-N 187 Uhrzeit: Mittwoch 12-14 Uhr \\ } % Declares the document's title.
\author{Moritz Schäfer (350651), \\
Marco Morik (348689), Max Gotthardt(350464)} % Declares the author's name.
\date{June 19 2014} % Deleting this command produces today's date.

\maketitle



\newpage
\section*{Hausaufgabe 8}
\subsection*{Aufgabe 1:}
\subsubsection*{1.a):}
1:
\begin{align*}
\auswf{F_1} &= \ausws{a} ( \auswl{a} \emptyset \cap \auswl{b} \emptyset \cap \auswl{c} \emptyset ) \cup \ausws{b} (\auswl{a} \emptyset \cap \auswl{b} \cap \auswl{c} \emptyset ) \cup \ausws{c}( \auswl{a} \emptyset \cap \auswl{b} \emptyset \cap \auswl{c} \emptyset )\\
&= \ausws{a} \{p_7\} \cup \ausws{b} \{p_7\} \cup \ausws{c} \{p_7\} \\
&= \{p_5\} \cup \{p_6\} \cup \{p_4\}  \\
&= \{ p_5, p_6, p_4 \}
\end{align*}
\\
2:
\begin{align*}
\auswf{F_2} &= \ausws{a} \ausws{a} Proc \cap \ausws{a} \ausws{b} Proc \\
&= \ausws{a} \{p_1,p_2,p_3,p_5,q_1,q_2,q_3,q_5,q_7,q_8,q_9 \} \cap \ausws{a} \{p_1,p_4,p_5,p_6,q_1,q_3,q_5,q_6,q_9,q_10 \} \\
&= \{p_3,q_1,q_2,q_3,q_7,q_8\} \cap \{p_1,p_2,q_1,q_2,q_3,q_7\} \\
&= \{q_1,q_2,q_3,q_7 \}
\end{align*}
\\
3:
\begin{align*}
\auswf{F_3} &= \auswl{b}( \ausws{c} \auswl{a} \emptyset \cup \ausws{b} \auswl{a} \emptyset) \\
&= \auswl{b}(\ausws{c}\{p_4,p_6,p_7,q_4,q_6,q_10\} \cup \ausws{b} \{p_4,p_6,p_7,q_4,q_6,q_10 \} ) \\
&= \auswl{b}( \{p_4,p_5,q_4,q_6,q_7\} \cup \{p_4,p_6,q_5,q_6\}) \\
&= \auswl{b} \{p_4,p_5,p_6,q_4,q_5,q_6,q_7 \} \\
&= \{p_2,p_3,p_4,p_7,q_2,q_4,q_5,q_7,q_8\}
\end{align*}
\\
4:
\begin{align*}
\auswf{F_4} &= \auswl{c} \ausws{a} \{p_1,p_2,p_3,p_5,q_1,q_2,q_3,q_5,q_7,q_8,q_9 \} \cup \ausws{c} \{p_1,p_4,p_5,p_6,q_1,q_3,q_5,q_6,q_9,q_10 \}  \\
&= \auswl{c} \{p_3,q_1,q_2,q_7,q_8\} \cup \{p_2,p_5,q_2,q_4,q_7 \} \\
&= \{p_1,p_3,p_7,q_1,q_2,q_3,q_5,q_8,q_9,q_10\} \cup \{p_2,p_5,q_2,q_4,q_7 \} \\
&= \{p_1,p_2,p_3,p_5,p_7,q_1,q_2,q_3,q_4,q_5,q_7,q_8,q_9,q_{10}\}
\end{align*}
\newpage
\subsection*{Aufgabe 2:}
\subsubsection*{a)}
i:\\
$F_i = \langle auflegen \rangle \langle warte1min \rangle \langle warte1min \rangle \langle warte1min \rangle \langle warte1min \rangle \langle wenden \rangle \langle warte1min \rangle \langle warte1min \rangle \langle warte1min \rangle \langle warte1min \rangle \langle essen \rangle tt $ \\
\\
ii:\\
$F_{ii} = \langle marinieren \rangle [marinieren] ff$\\
Die Formel besagt nur, dass nicht 2 mal direkt hintereinander mariniert wird. Mit einer Aktion dazwischen kann immer noch 2fach mariniert werden \\
\\
iii: \\
$F_{iii} = \langle wenden \rangle \langle warte1min \rangle \langle wenden \rangle tt $ \\
\\ \\ 
iv: 
\begin{align*}
~\\[-25pt]
F_{iv} &= ( \langle wenden \rangle \langle warte1min \rangle \langle wenden \rangle tt )^c \\
&= [wenden] [warte1min] [wenden] ff
\end{align*}
\\
v:
\begin{align*}
~\\[-25pt]
F_{v} =& [essen][loeschen] ff \\
&\wedge [auflegen] ( \langle essen \rangle \langle loeschen \rangle tt \vee \langle warten \rangle tt ) \\
&\wedge [warte1min] ( \langle essen \rangle \langle loeschen \rangle tt \vee \langle warten \rangle tt ) \\
&\wedge [wenden] ( \langle essen \rangle \langle loeschen \rangle tt \vee \langle warten \rangle tt )  \\
 =& [essen][loeschen] ff \wedge [\{auflegen,warte1min,wenden\}] ( \langle essen \rangle \langle loeschen \rangle tt \vee \langle warten \rangle tt )
\end{align*}
\subsubsection*{b)}
\begin{align*}
\auswf{F_i} &= \{GuF,N\} \\
\auswf{F_{ii}} &= Proc \\
\auswf{F_{iii}} &= \{W \} \\
\auswf{F_{iv}} &= Proc \setminus \{W\} \\
\auswf{F_{v}} &= Proc \setminus \{S\}
\end{align*}
\\
i:\\
Ja, da $GuF \in \auswf{F_i}$\\ \\
ii: \\
Ja, da Zustand $w \models F_{iii}$ bzw. $w \in \auswf{F_{iii}}$\\ \\
iii: \\
Nein, da Zustand  $S \nvDash F_v $ bzw. $S \notin \auswf{F_v} $ \\
Somit gilt dies nicht für Alle Zustände 
\newpage
\subsection*{Aufgabe 3:}


  \begin{minipage}{0.4\linewidth-7.112pt}
   \makebox[\linewidth]{
    \begin{xy}
      \xymatrix
      {
       & \Pro{p_1} \ar[dr]^{\In{c}{}} \ar[dl]^{\In{a}{}} & \\
       \ar @(ul,dl)_{\In{a}{}} \Pro{p_2} \ar[dr]^{\In{b}{}}  & & \Pro{p_4} \\
       & \Pro{p_3} \ar[ur]^{\In{b}{}}& 
      }
    \end{xy}
   }
  \end{minipage}
\newpage
\section*{Hausaufgabe 9}
\subsection*{Aufgabe 4:}
p und q sind nicht stark bisimilar, es gibt die unterscheidende Formel $F_1$ : \\
$F_1 = \langle a \rangle \langle a \rangle \langle b \rangle tt $ \\
$q \in \auswf{F_1}$ und $p \notin \auswf{F_1}$\\
\\
\\
r und s sind nicht stark bisimilar, es gibt die unterscheidende Formel $F_2$ : \\
$F_2 = \langle a \rangle [b] ff $ \\
$r \in \auswf{F_2}$ und $s \notin \auswf{F_2}$\\
\\
\\
u und v sind nicht stark bisimilar, es gibt die unterscheidende Formel $F_3$ : \\
$F_3 = \langle a \rangle (\langle b \rangle tt \wedge \langle c \rangle tt ) $\\
$v \in \auswf{F_3}$ und $u \notin \auswf{F_3}$\\
\subsection*{Aufgabe 5:}
P und Q sind nicht stark bisimilar, es gibt die unterscheidende Formel $F_1$ : \\
$F_1 = \langle a \rangle \langle a \rangle tt$ \\
$ P\in \auswf{F_1}$ und $Q \notin \auswf{F_1}$\\
\\
\\
U und V sind stark bisimilar, es gibt keine unterscheidende Formel $F_2$ : \\
\\
\\
X und Y sind nicht stark bisimilar, es gibt die unterscheidende Formel $F_3$ : \\
$F_3 =  [a] \langle b \rangle tt$ \\
$X \in \auswf{F_3}$ und $Y \notin \auswf{F_3}$\\
\\
\\
$S_n$ und T sind nicht stark bisimilar, es gibt die unterscheidende Formel $F_4$ : \\
$F_4 = \langle a_{10} \rangle \langle a_1 \rangle tt $ \\
$T \in \auswf{F_4}$ und $S_n \notin \auswf{F_4}$\\
\\
\\
\subsection*{Aufgabe 6:}
z.Z. P$\bisim _n Q$ für alle n$\in \mathbb{N} \implies P \bisim Q $ \\
Annahme: $P \bisim _n Q $ für alle n$\in \mathbb{N}$\\
z.Z. $P\bisim Q$\\
$B= \{ (R,S) | R,S \in Proc \wedge R \bisim _n S$ für alle $n\in \mathbb{N} \} $\\ \\
Wiederspruch: $R \CCSTrans {\alpha} R'$ , aber $ \neg \exists S \CCSTrans {\alpha}S'.(R',S')\in B$ für $\alpha\in \Act$ beliebig aber fest \\
Da T bild-endlich gilt: $\#(Der(S,\alpha))\in \mathbb{N}$, also $\{S'_1,...,S'_n\}$ \\
$\forall i \in \{1,...,n\}$, gibt es ein $p_i \in \mathbb{N}$ mit $ \neg (R' \bisim _{p_i} S_i')$ \\
Daraus folgt, das es ein $p_{max}= max(\{p_1,...p_n\})+1$ gibt für das  gilt: \\
$ \neg( R \bisim _{p_{max}} S)$ (Da  sowohl $R \CCSTrans {\alpha} R'$ , als auch $S\CCSTrans {\alpha}S' $) \\
WIEDERSPRUCH!!!\\
Somit muss gelten $P\bisim Q $\\

\newpage
\section*{Hausaufgabe 10:}
\subsection*{Aufgabe 7:}
Zuerst zeigen wir \\
\\
Induktions-Anfang: \\
\\
$\OSem{X}{S} = S$  \\
$\OSem{X}{A} = A$ , wenn $S \subseteq A$ ,folgt daraus $\OSem{X}{S} \subseteq \OSem{X}{A}$ , daher ist $\OSem{X}{}$ monoton\\ \\
$\OSem{\true}{S} = \Proc$  \\
$\OSem{\true}{A} = \Proc$ , da $\Proc \subseteq \Proc$, folgt daraus $\OSem{\true}{}$ ist monoton \\ \\
$\OSem{\false}{S} = \emptyset$ \\
$\OSem{\false}{A} = \emptyset$ ,da $\emptyset \subseteq \emptyset$, folgt daraus $\OSem{\false}{}$ ist monoton \\ \\
InduktionsAnnahme:
$\OSem{F}{A}$ ist monoton \\ \\
InduktionsSchritt:\\
\textbf{Disjunktion:} \\
$\OSem{F \lor G}{S} = \OSem{F}{S} \cup \OSem{G}{S}$ \\
$\OSem{F \lor G}{A} = \OSem{F}{A} \cup \OSem{G}{A}$ \\ \\
Da Annahme $\OSem{F}{S} \subseteq \OSem{F}{A}$  \\
Da Annahme $\OSem{G}{S} \subseteq \OSem{G}{A}$  \\ \\
Somit ist auch die Vereinigung eine Teilmenge \\
$\OSem{F}{S} \cup \OSem{G}{S} \subseteq \OSem{F}{A} \cup \OSem{G}{A}$ \\
Daraus folgt, dass die Disjunktion $\OSem{F \lor G}{}$ monton ist  \\ \\ 
\textbf{Bedingung:} \\
$\OSem{[a]}{S} = \auswl{a}\OSem{F}{S}$ , und  $\auswl{a} \OSem{F}{S} = \{p \in \Proc \mid \forall p' \in Der(p,a).p' \in \OSem{F}{S}\}$ \\
$\OSem{[a]}{A} = \auswl{a}\OSem{F}{A}$ , und $\auswl{a}\OSem{F}{A} = \{p \in \Proc \mid \forall p' \in Der(p,a).p' \in \OSem{F}{A}\}$ \\
Da $\OSem{F}{S} \subseteq \OSem{F}{A}$, folgt : \\
\begin{align*}
\auswl{a}\OSem{F}{S} &= \{p \in \Proc \mid \forall p' \in Der(p,a).p' \in \OSem{F}{S}\} \\
 & \subseteq \auswl{a}\OSem{F}{A} = \{p \in \Proc \mid \forall p' \in Der(p,a).p' \in \OSem{F}{A}\} \\
 &=\auswl{a}\OSem{F}{A}
\end{align*}
Daher ist auch die Bedingung $\OSem{[a]}{}$ monoton

\subsection*{Aufgabe 8:}
\subsubsection*{a)}
$F_i = X$ \\
$X \HMmax [\Act ] X \wedge (\langle a \rangle \true \vee \langle b \rangle \true $)
\subsubsection*{b)}
$F_{ii} = \langle c \rangle \true \wedge [a]X$ \\
$X \HMmin \langle \Act \rangle X \vee [a] \false \vee \langle b \rangle \true $

\subsection*{Aufgabe 9:}
\subsubsection*{a)}
$\fix \OSem{(\langle a \rangle \true \vee \langle b \rangle \true ) \wedge [\Act ]X}{}= (\OSem {(\langle a \rangle \true \vee \langle b \rangle \true )\wedge [\Act ]X )}{\emptyset})^M $ 
\begin{align*}
(\OSem {(\langle a \rangle \true \vee \langle b \rangle \true )\wedge [\Act ]X )}{\emptyset})^1 & \overset{\OSem{F}{}}{=} (\ausws{a}\Proc \cup \ausws{b} \Proc ) \cap \auswl{\Act}\emptyset \\
&\overset{\ausws{} \auswl{} }{=} (\{p_1,p_2,p_5\} \cup \{p_1,p_3,p_4\} ) \cap \{p_6\} \\
&\overset{\cdot \cup \cdot}{=} \{p_1,p_2,p_3,p_4,p_5\} \cap \{p_6\} \\
&\overset{\cdot \cap \cdot}{=} \emptyset
\end{align*}
Die leere Menge ist der kleinste Fixpunkt der Formel
\subsubsection*{b)}
$\FIX \OSem{(\langle a \rangle \true \vee \langle b \rangle \true ) \wedge [\Act ]X}{}= (\OSem {(\langle a \rangle \true \vee \langle b \rangle \true )\wedge [\Act ]X )}{\Proc})^M $ 
\begin{align*}
(\OSem {(\langle a \rangle \true \vee \langle b \rangle \true )\wedge [\Act ]X )}{\Proc})^1 & \overset{\OSem{F}{}}{=} (\ausws{a}\Proc \cup \ausws{b} \Proc ) \cap \auswl{\Act}\Proc \\
&\overset{\ausws{} \auswl{} }{=} (\{p_1,p_2,p_5\} \cup \{p_1,p_3,p_4\} ) \cap \{p_1,p_2,p_3,p_4,p_5,p_6\} \\
&\overset{\cdot \cup \cdot}{=} \{p_1,p_2,p_3,p_4,p_5\} \cap \{p_1,p_2,p_3,p_4,p_5,p_6\} \\
&\overset{\cdot \cap \cdot}{=} \{p_1,p_2,p_3,p_4,p_5\}
\end{align*}
\begin{align*}
(\OSem {(\langle a \rangle \true \vee \langle b \rangle \true )\wedge [\Act ]X }{\Proc})^2 & \overset{\OSem{F}{}}{=} (\ausws{a}\Proc \cup \ausws{b} \Proc ) \cap \auswl{\Act}  \{p_1,p_2,p_3,p_4,p_5\}\\
&\overset{\ausws{} \auswl{} }{=} (\{p_1,p_2,p_5\} \cup \{p_1,p_3,p_4\} ) \cap \{p_1,p_2,p_3,p_5,p_6\} \\
&\overset{\cdot \cup \cdot}{=} \{p_1,p_2,p_3,p_4,p_5\} \cap \{p_1,p_2,p_3,p_5,p_6\} \\
&\overset{\cdot \cap \cdot}{=} \{p_1,p_2,p_3,p_5\}
\end{align*}
\begin{align*}
(\OSem {(\langle a \rangle \true \vee \langle b \rangle \true )\wedge [\Act ]X }{\Proc})^3 & \overset{\OSem{F}{}}{=} (\ausws{a}\Proc \cup \ausws{b} \Proc ) \cap \auswl{\Act}  \{p_1,p_2,p_3,p_5\}\\
&\overset{\ausws{} \auswl{} }{=} (\{p_1,p_2,p_5\} \cup \{p_1,p_3,p_4\} ) \cap \{p_2,p_5,p_6\} \\
&\overset{\cdot \cup \cdot}{=} \{p_1,p_2,p_3,p_4,p_5\} \cap \{p_2,p_5,p_6\} \\
&\overset{\cdot \cap \cdot}{=} \{p_2,p_5\}
\end{align*}
\begin{align*}
(\OSem {(\langle a \rangle \true \vee \langle b \rangle \true )\wedge [\Act ]X}{\Proc})^4 & \overset{\OSem{F}{}}{=} (\ausws{a}\Proc \cup \ausws{b} \Proc ) \cap \auswl{\Act}  \{p_2,p_5\}\\
&\overset{\ausws{} \auswl{} }{=} (\{p_1,p_2,p_5\} \cup \{p_1,p_3,p_4\} ) \cap \{p_2,p_6\} \\
&\overset{\cdot \cup \cdot}{=} \{p_1,p_2,p_3,p_4,p_5\} \cap \{p_2,p_6\} \\
&\overset{\cdot \cap \cdot}{=} \{p_2\}
\end{align*}
\begin{align*}
(\OSem {(\langle a \rangle \true \vee \langle b \rangle \true )\wedge [\Act ]X }{\Proc})^5 & \overset{\OSem{F}{}}{=} (\ausws{a}\Proc \cup \ausws{b} \Proc ) \cap \auswl{\Act}  \{p_2\}\\
&\overset{\ausws{} \auswl{} }{=} (\{p_1,p_2,p_5\} \cup \{p_1,p_3,p_4\} ) \cap \{p_2,p_6\} \\
&\overset{\cdot \cup \cdot}{=} \{p_1,p_2,p_3,p_4,p_5\} \cap \{p_2,p_6\} \\
&\overset{\cdot \cap \cdot}{=} \{p_2\}
\end{align*}
Die Menge $\{p_2\}$ ist der größte Fixpunkt der Formel $(\langle a \rangle \true \vee \langle b \rangle \true )\wedge [\Act ]X $\\
\end{document}
